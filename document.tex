%! TEX program = xelatex
\documentclass{ctexart}
\usepackage{listings}
\usepackage{color}
\usepackage{float}
\usepackage{geometry}
\geometry{a4paper,scale=0.8}
\usepackage{amsmath}
\usepackage{amsfonts}
\usepackage{amssymb}
\title{\LARGE\heiti {华罗庚协会讨论班题目}}
\author{华仔 }
\date{}

\begin{document}

\maketitle

\section{第一周}

\subsection{课上题目}

1.1: 设$a,b\in\mathbb{N}^{+}$,$\frac{a+1}{b}+\frac{b+1}{a}\in \mathbb{Z}$,证明:
	
\[
(a,b)\le\sqrt{a+b}
\]
\rightline{\footnotesize wzl供题}

1.2:$a,b\in \mathbb{N}^{+},(a,b)=1$,证明:$(a+b,a^2+b^2)$为$1$或者$2$

    \rightline{\footnotesize wzl供题}
    
1.3: $n\ge m>0$,证明$a$为正整数,其中

\[a=\frac{(m,n)}{n}·C_{n}^{m}
    \]
\rightline{\footnotesize wzl供题}

1.4:$p,q$为素数,$q=p+2$,证明:$p+q|p^q+q^p$
	
		\rightline{\footnotesize wzl供题}

1.5:$m,n\in \mathbb{N}^{+},m>n$,证明:

\[
\left[m,n\right]+\left[m+1,n+1\right]>\frac{2mn}{\sqrt{m-n}}
      \]
    \rightline{\footnotesize wzl供题}

1.6:试证明:任意大于$2$的偶数,可以写成两个无平方因子数之和.

\rightline{\footnotesize 学长供题}


\subsection{未讲完的题目}
1.0: 证明不等式$[\sqrt{\alpha}]+[\sqrt{\alpha+\beta}]+[\sqrt{\beta}]\ge [\sqrt{2 \alpha}]+[\sqrt{2\beta}]$对任意不小于1的实数$\alpha$和$\beta$成立
\rightline{\footnotesize wzl供题}

1.1:设$a,b,c,d$为整数,$(a-c)|(ab+cd)$,则:

	\[
	(a-c)|(ad+bc)
	\]
		\rightline{\footnotesize wzl供题}

1.2:设$a,b$都是正整数,$a^2+ab+1$被$b^2+ab+1$整除,证明:

	\[
	a=b
	\]
	\rightline{\footnotesize wzl供题}

1.3:证明存在无穷多个正整数$n$,使得
	
		\[
		n|(2^n+2),(n-1)|(2^n+1)
		\]
		\rightline{\footnotesize wzl供题}

1.4:设$p$是素数,$x,y,z\in \mathbb{Z}$满足$0<x<y<z<p$,$x^3\equiv y^3\equiv z^3 (mod p)$,证明:
	
		\[
		(x+y+z)|(x^2+y^2+z^2)
		\]
		\rightline{\footnotesize wzl供题}


1.5: 给定$C>0$,对$n=\displaystyle\prod_{i=1}^{n} P_{i}^{\alpha_{i}}$,定义$\mho(n)=\displaystyle\sum_{p_i>C} \alpha_{i}$,求:
$\Phi :z \rightarrow z$使得对于$\forall a,b\in \mathbb{N^{+}},a>b$有:

\[\mho (\Phi (a)-\Phi (b))\le \mho (a-b)\]

\rightline{\footnotesize 学长供题}


\end{document}