%! TEX program = xelatex
\documentclass{ctexart}
\usepackage{listings}
\usepackage{color}
\usepackage{float}
\usepackage{geometry}
\geometry{a4paper,scale=0.8}
\usepackage{amsmath}
\usepackage{amsfonts}
\usepackage{amssymb}
\title{\LARGE\heiti {华罗庚协会讨论班部分题目答案}}
\author{华仔 }
\date{}

\begin{document}

\maketitle

\section{第一周}
1.1:设$a,b,c,d$为整数,$(a-c)|(ab+cd)$,则:

	\[
	(a-c)|(ad+bc)
	\]
		\rightline{\footnotesize wzl供题}

提示:$(ab+cd)-(ad+bc)=(a-c)(b-d)$

1.2:设$a,b$都是正整数,$a^2+ab+1$被$b^2+ab+1$整除,证明:

	\[
	a=b
	\]
	\rightline{\footnotesize wzl供题}

提示:
\[b^2+ab+1-
(a^2+ab+1)=(a+b)(a-b)\Rightarrow a^2+ab+1|(a+b)(a-b)\]

1.3:设$p$是素数,$x,y,z\in \mathbb{Z}$满足$0<x<y<z<p$,$x^3\equiv y^3\equiv z^3 (mod p)$,证明:
	
		\[
		(x+y+z)|(x^2+y^2+z^2)
		\]
		\rightline{\footnotesize wzl供题}

解答:$p|x^3-y^3,p\nmid |x-y|\Rightarrow$
\[p|x^2+xy+y^2\]


同理,$p|y^2+yz+z^2$
\[p|x^2+xy+y^2-y^2-yz-z^2\Rightarrow p|(x-z)(x+y+z)\Rightarrow p|x+y+z\]


由条件$0<x<y<z<p$,所以$x+y+z=p,2p$,由于$p>3$,$(2,p)=1$
\[x+y+z\equiv x^2+y^2+z^2(mod2)\]


只需证$p|x^2+y^2+z^2$,由于$p|x(x+y+z)+y^2-xz$
\[p|y^2-xz,p|x^2-yz,p|z^2-xy\]


所以$p|3(x^2+y^2+z^2)$,而$p\nmid 3$,即得结论



1.4:证明存在无穷多个正整数$n$,使得
	
		\[
		n|(2^n+2),(n-1)|(2^n+1)
		\]
		\rightline{\footnotesize wzl供题}

解答:

由于$2|2^2+2,2-1|2^2+1$,所以存在$n$,假设$\exists n$,使得题目成立,

\[\Rightarrow 2|n,4\nmid n\Rightarrow \exists R,2^n+1=R(n-1)\]

由于$2\nmid n-1,2\nmid R\Rightarrow $

\[2^{2^n+1}+1=2^{R(n-1)}+1=(2^{n-1}+1)m\]

所以$2^{n-1}+1|2^{2^n+1}+1\Rightarrow (2^n+2)|(2^{2^n+2}+2)$,由于$n|2^n+2$且$4\nmid n $,所以
\[2^n+2=nt\]

$t$是奇数,所以$2^{2^n+2}+1=2^{nt}+1=(2^n+1)s,s\in \mathbb{N}^{+}\Rightarrow 2^n+1|2^{2^n+2}+1$,成立
\end{document}
